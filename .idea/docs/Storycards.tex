\documentclass[12pt]{article}
\usepackage[utf8]{inputenc}
\usepackage[a4paper,margin=16pt]{geometry}
\usepackage{tcolorbox}
\usepackage{lmodern}
\usepackage{ragged2e}

\pagestyle{empty}

\newtcolorbox{storycard}{
  colback=white,
  colframe=black,
  width=\textwidth,
  height=0.16\textheight,  % Smaller height for 5 cards
  boxrule=0.8pt,
  sharp corners,
  valign=center,
  halign=left,
}

\newcommand{\storycardcontent}[6]{%
  \begin{storycard}
    \noindent
    \begin{minipage}[t]{0.32\textwidth}
      \textbf{Priorisierung:} #1
    \end{minipage}
    \hfill
    \begin{minipage}[t]{0.32\textwidth}
      \centering
      \textbf{Name der Story:} #2
    \end{minipage}
    \hfill
    \begin{minipage}[t]{0.32\textwidth}
      \raggedleft
      \textbf{Story Points:} #3
    \end{minipage}

    \vfill

    \begin{center}
      \textbf{Als #4 möchte ich #5, um #6.}
    \end{center}

    \vfill

    \noindent
    \begin{minipage}[t]{0.49\textwidth}
      \textbf{Risiko:} Niedrig
    \end{minipage}
    \hfill
    \begin{minipage}[t]{0.49\textwidth}
      \raggedleft
      \textbf{Story Points (Post-Schätzung):} --
    \end{minipage}
  \end{storycard}

  \vspace{8pt} % Space between cards
}

\begin{document}

\storycardcontent{Hoch}{Platzierung von Objekten}{3}{Nutzer}{verschieden Objekte platzieren können, welche unterschiedliche Physics haben}{das Spiel zu spielen und zu genießen}

\storycardcontent{Hoch}{Button für Simulation}{1}{Nutzer}{einen Button haben mitdem ich die Simulation starten/abbrechen und auch Zurücksetzen kann}{in den Editiermodus zurückzukommen}

\storycardcontent{Hoch}{Editier Modus}{4}{Nutzer}{einen Editier Modus haben in welchem ich Objekte platzieren/entfernen kann}{eigene Level zu bauen}

\storycardcontent{Mittel}{Support Level Dateien}{2}{Nutzer}{mein gebautes Level als Datei exportieren können}{es mit anderen Teilen zu können}

\storycardcontent{Niedrig}{Level Dateien Menschenlesbar}{1}{Nutzer}{die Level-Dateien lesen können}{}

\storycardcontent{Hoch}{Level Progression}{4}{Nutzer}{nach Abschließen eines Levels das Gefühl haben eine Level Progression zu erkennen}{kein Langweiliges Spiel zu haben}

\storycardcontent{Hoch}{Simple Grafik}{1}{Nutzer}{simple aber erkennbare Objekte haben}{diese gut nutzen zu können}

\storycardcontent{Mittel}{Farben}{1}{Nutzer}{das nicht alles Schwarz/Weiß ist sondern schön farbig, also alle Objekte und Level}{mehr Spielspaß zu haben und das Gefühl ein modernes Spiel zu spielen}

\storycardcontent{Hoch}{Echtzeit Anzeige}{2}{Nutzer}{das mir alles in Echtzeit angezeit wird und keine lange Wartezeit zwischen Aktionen ist}{den Spielspaß nicht zu verlieren}

\storycardcontent{Hoch}{Win Conditions}{2}{Nutzer}{Win Conditions haben welche sich gut erkennen lassen bzw. gut erklärt sind}{zu vermeiden das gleiche Level mehrfach Spielen zu müssen bis man das Ziel des Level sieht/versteht}

\storycardcontent{Mittel}{Einschränkungszonen}{3}{Nutzer}{gut erkennbar sehen wo ich Objekte platzieren darf}{}

\storycardcontent{Mittel}{Inventar}{3}{Nutzer}{ein Inventar haben indem alle Objekte sind welche ich platzieren kann}{besser planen zu können wie ein Level zu lösen ist}

\storycardcontent{Priorität}{Name}{Points}{Rolle}{Wunsch}{Nutzen}
\end{document}
\documentclass[12pt]{article}
\usepackage[utf8]{inputenc}
\usepackage[a4paper,margin=16pt]{geometry}
\usepackage{tcolorbox}
\usepackage{lmodern}
\usepackage{ragged2e}

\pagestyle{empty}

\newtcolorbox{storycard}{
  colback=white,
  colframe=black,
  width=\textwidth,
  height=0.16\textheight,  % Smaller height for 5 cards
  boxrule=0.8pt,
  sharp corners,
  valign=center,
  halign=left,
}

\newcommand{\storycardcontent}[6]{%
  \begin{storycard}
    \noindent
    \begin{minipage}[t]{0.32\textwidth}
      \textbf{Priorisierung:} #1
    \end{minipage}
    \hfill
    \begin{minipage}[t]{0.32\textwidth}
      \centering
      \textbf{Name der Story:} #2
    \end{minipage}
    \hfill
    \begin{minipage}[t]{0.32\textwidth}
      \raggedleft
      \textbf{Story Points:} #3
    \end{minipage}

    \vfill

    \begin{center}
      \textbf{Als #4 möchte ich #5, um #6.}
    \end{center}

    \vfill

    \noindent
    \begin{minipage}[t]{0.49\textwidth}
      \textbf{Risiko:} Niedrig
    \end{minipage}
    \hfill
    \begin{minipage}[t]{0.49\textwidth}
      \raggedleft
      \textbf{Story Points (Post-Schätzung):} --
    \end{minipage}
  \end{storycard}

  \vspace{8pt} % Space between cards
}

\begin{document}

\storycardcontent{Hoch}{Login}{5}{Nutzer}{mich einloggen}{mein Dashboard zu sehen}
\storycardcontent{Mittel}{Passwort zurücksetzen}{3}{Nutzer}{mein Passwort zurücksetzen}{Zugang zu erhalten}
\storycardcontent{Niedrig}{Profil bearbeiten}{2}{Nutzer}{mein Profil ändern}{meine Daten zu pflegen}
\storycardcontent{Hoch}{Suche verwenden}{4}{Nutzer}{nach Produkten suchen}{schneller finden}
\storycardcontent{Mittel}{Benachrichtigungen konfigurieren}{3}{Nutzer}{Benachrichtigungen steuern}{nicht gestört zu werden}

\end{document}
